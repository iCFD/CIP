\documentclass[10pt,a4paper]{article}
\usepackage{latexsym}
\usepackage{amsmath,amsfonts,amssymb,amscd,amsthm,xspace}  
\usepackage{natbib}
\usepackage{lineno}

% Page length commands go here in the preamble
\setlength{\oddsidemargin}{0.10in}	% Left margin of 1 in + 0 in = 1 in
\setlength{\textwidth}{6.50in}		% Right margin of 8.5 in - 1 in - 6.5 in = 1 in
\setlength{\topmargin}{-0.75in}		% Top margin of 2 in -0.75 in = 1 in
\setlength{\textheight}{9.2in} 		% Lower margin of 11 in - 9 in - 1 in = 1 in

% For Theorems and Definitions
\newtheorem{lemma}{Lemma}
\newtheorem{remark}{Remark}

% Manuel's Macros
\renewcommand{\u}{\boldsymbol{\upsilon}} % for vector upsilon
\newcommand{\p}{\boldsymbol{p}} 		% for vector p
\renewcommand{\r}{\boldsymbol{r}} 		% for vector r
\newcommand{\x}{\boldsymbol{x}} 		% for vector x
\renewcommand{\c}{\boldsymbol{c}} 		% for vector c
\newcommand{\vecxi}{\boldsymbol{\xi}}	% for vector xi
\DeclareMathOperator{\Sample}{Sample}
\let\vaccent=\v % rename builtin command \v{} to \vaccent{}
\renewcommand{\v}[1]{\ensuremath{\mathbf{#1}}} % for vectors
\newcommand{\gv}[1]{\ensuremath{\mbox{\boldmath$ #1 $}}} 
% for vectors of Greek letters
\newcommand{\uv}[1]{\ensuremath{\mathbf{\hat{#1}}}} % for unit vector
\newcommand{\abs}[1]{\left| #1 \right|} % for absolute value
\newcommand{\avg}[1]{\left< #1 \right>} % for average
\let\underdot=\d % rename builtin command \d{} to \underdot{}
\renewcommand{\d}[2]{\frac{d #1}{d #2}} % for derivatives
\newcommand{\dx}[1]{\frac{d}{d #1}} % for derivatives without specifying the derived function
\newcommand{\dd}[2]{\frac{d^2 #1}{d #2^2}} % for double derivatives
\newcommand{\pd}[2]{\frac{\partial #1}{\partial #2}} 
% for partial derivatives
\newcommand{\pdd}[2]{\frac{\partial^2 #1}{\partial #2^2}} 
% for double partial derivatives
\newcommand{\pdc}[3]{\left( \frac{\partial #1}{\partial #2}
 \right)_{#3}} % for thermodynamic partial derivatives
\newcommand{\ket}[1]{\left| #1 \right>} % for Dirac bras
\newcommand{\bra}[1]{\left< #1 \right|} % for Dirac kets
\newcommand{\braket}[2]{\left< #1 \vphantom{#2} \right|
 \left. #2 \vphantom{#1} \right>} % for Dirac brackets
\newcommand{\matrixel}[3]{\left< #1 \vphantom{#2#3} \right|
 #2 \left| #3 \vphantom{#1#2} \right>} % for Dirac matrix elements
\newcommand{\grad}[1]{\gv{\nabla} #1} % for gradient
\newcommand{\gradx}[2]{\gv{\nabla}_{#2} #1} % for gradient
\let\divsymb=\div % rename builtin command \div to \divsymb
\renewcommand{\div}[1]{\gv{\nabla} \cdot #1} % for divergence
\newcommand{\divx}[2]{\gv{\nabla}_{#2} \cdot #1} % for divergence v2
\newcommand{\curl}[1]{\gv{\nabla} \times #1} % for curl

% Spacing and Indentation
\renewcommand{\baselinestretch}{1.5} % 1.5 denotes double spacing. Changing it will change the spacing
%\setlength{\parindent}{0in} 

%%%%%%%%%%%%%%%%%%%
% Document begins!
%%%%%%%%%%%%%%%%%%%

\begin{document}
\title{Conservative Semi-Lagrangian Reconstructions}
\author{Oleg Kravchenko\textsuperscript{1} \& Manuel Diaz* \\ 
\small \textsuperscript{1} Department of Higher Mathematics, Bauman Moscow State Technical University, Moscow, Russia. \\[-3mm]
\small * Institute of Applied Mechanics, National Taiwan University, Taiwan R.O.C.}
\date{\today}

\maketitle

\linenumbers

\abstract{A class of high-order constrained reconstructions, in the context of finite difference methods for scalar hyperbolic conservation laws, are presented and numerically studied in the presence of local extrema, or shock solutions.
Although these reconstructions can use large time steps given their semi-Lagrangian in nature, a conservative formulation of the conservation is here considered. The numerical one-dimensional implementations in Matlab are reported to provide a simple follow up of the formulations here versed. Application to the nonlinear Boltzmann-BGK equation to model the dynamics of inert gas flows in near continuum regimes are explored.}

\section{Introduction}

%%%%%%%%%%%%%%
% NOTES:
%%%%%%%%%%%%%%
%
% **Note that nothing here right now if considered final. The purpose of this document is simply to serve as an enviroment (or a very orginized board) meant to collaborate/play/explore/test CIP formulations and any new idea that comes in time. Manuel Diaz. 2015/07/28.
%
% **Use "rsync -r /home/manuel/Dropbox/Apps/Texpad/Paper3/CIP/ /home/manuel/github/CIP/Tex/" to sincronize the Dropbox version with the GitHub version. (Debian 7 only!) Manuel Diaz. 2015/07/28.
%
%

% Constext of the problem/challenge/idea
The semi-Lagrangian method is a numerical solution technique for the partial differential equations describing the advection process. It accounts for the Lagrangian nature of the transport process but, at the same time, it allows to work on a fixed computational grid. Starting from the first proposals to in the meteorological literature, which focused on the advection of vorticity in simplified models of large scale flow, it has developed into a mature discretization approach for the complete equations of atmospheric flows. The semi-Lagrangian is also related (and, in some cases, entirely equivalent) to similar methods developed in other modelling communities, such as for example the modified method of characteristics, the Eulerian-Lagrangian method, the characteristic Galerkin method and the constrained interpolation polynomial methods. This last, have been proved to achieve excellent results for scalar conservation laws.
%
% I'll change the above. I took it from: http://www1.mate.polimi.it/~bonavent/Homepage_Luca_file/lezlag000.pdf

% What has been done?
Much work has been done in the text of semi-lagrangian constrained reconstructions with polynomials, here we point some of our observations:
\begin{itemize}
	\item \cite{Yabe1991} introduced the initial concept of cubic interpolation polynomial (CIP) for the evolution of scalar advection equations considering the also the evolution of the cell averages slope information. This simple idea has been proved to be workable but conservation cannot be guaranteed.
	\item Two years later \cite{Yabe2001}, introduced a new reformulation of their CIP algorithm which now considers the cell numerical-mass conservation as a constrain and a reconstruction of a flux as and integral function is introduced in this semi-Lagrangian formulation.
	\item Conservative formulation of CIP methods where first introduced by \cite{Tanaka2000} and \cite{Xiao2001}.
	\item Extention to multi-dimensional formulation of the CIP algorithms where reported by \cite{Nakamura2001}.
\end{itemize}

% What the solution procedure/proposal?
It is interesting to point out the conceptual similarities of CIP-CSL schemes in \cite{LiXiao2007} with the \emph{flux reconstruction} (FR) schemes by \cite{Huynh2007}. As both formulation achieve successfully high-order and accurate representations. However a clear advantage of the CIP-CSL is the simple incorporation of slope limiters in to the formulation which in the context of FR schemes breaks the abstraction of correction functions. Moreover, although their success the numerical experience of these methods reveals that both implementations require bast computational resources as their stability conditions can only allow small time steps. 
The above is also result of the fact that CIP-CSL and FR are closer to an Eulerian description of flow conditions, see for example \cite{LiXiao2009}.

It is found that CIP literature is rich in ideas and strategies to ensure the conservative properties of the method, improve their accuracy or deal with the oscillations that arise near local extrema. We notice that CIP implementation with rational functions by \cite{Xiao1996}. Here the use of such function instead of polynomials led to limiter free CIP second-order implementation that can deal with discontinuities in the solution in a straight manner. This observation resonates with the \emph{local dual logarithmic reconstructions} (LDLR) by \cite{Artebrant2006}. Which led us to consider a non-polynomial reconstructions to formulate new semi-Lagrangian reconstruction that can handle discontinuities that arise with the solutions in a much straight and computationally economic manner.

% How is this document organized?
The present document is , section \ref{sec:CIP} introduces the basic formulation of CIP by \cite{Yabe1991}, followed by its modification to a fully conservative semi-Lagrangian scheme in section \ref{sec:CIP-CSL}. In section \ref{sec:CInonP-CSL} the concept of constrained interpolation is considered with a non-polynomial function. Lastly, some concluding remarks are given section \ref{sec:Conclusions}.

\section{Accuracy of Polynomial and non-Polynomial Interpolation Functions}

\begin{lemma} 
Consider a given function $f$ and denote $g$ as an approximation of $f$ such that $g\in C^3[a,b]$, with $\Delta=b-a$. Assuming that an integral relation of the form
%
\begin{linenomath}
\begin{equation}
	%
	\int_{a}^{b} f(x)dx = \int_{a}^{b} g(x)dx,
	%
\end{equation}
\end{linenomath}
%
hold and that the lateral gradient agree up to second-order
%
\begin{linenomath}
\begin{equation}
	%
	(f-g)'(a) = \mathcal{O}(\Delta x^2) = (f-g)'(b),
	\label{eq:lateralgradientsAssumption}
	%	
\end{equation}
\end{linenomath}
%
then, $(f-g)(x) = \mathcal{O}(\Delta x^3)$ for all $x\in[a,b]$.
\end{lemma}

\begin{proof}
We define $w(x)\equiv (f-g)(x)$. A Taylor expansion around $x'\in[a,b]$ yields
%
\begin{equation}
	%
	w(x) = w(x') + w'(x')(x-x') + \frac{1}{2}w''(x')(x-x')^2 + \mathcal{O}(|x-x'|^3),
	%
\end{equation}
%
for an arbitrary $x\in[a,b]$. By continuity of $f$, $g$ and because of $\int_{a}^{b} w(x)dx=0,$ we know that there exit a $x'\in[a,b]$, such that
%
\begin{equation}
	%
	w(x') = 0.
	%
\end{equation}
%
The mean value theorem implies that
%
\begin{equation}
	%
	w''(x') = \frac{w'(a)-w'(b)}{\Delta x},
	%
\end{equation}
%
for $x'\in[a,b]$ and $\Delta x=b-a$. With assumption (\ref{eq:lateralgradientsAssumption}) it follows that $w''(x') = \mathcal{O}(\Delta x)$. A Taylor expansion of $w'(x')$ around $a$ yields
%
\begin{equation}
	%
	w'(x') = x'(a) + \mathcal{O}(|x'-a|^2) = \mathcal{O}(\Delta x^2),
	%
\end{equation}
%
since $w'(a)=\mathcal{O}(\Delta x^2)$ and $w''(a) = \mathcal{O}(\Delta x)$. Consequently we get
%
\begin{equation}
	%
	w(x)\equiv(f-g)(x)=\mathcal{O}(\Delta x^3),
	%
\end{equation}
%
which concludes the proof. 
\end{proof}
%
Note that $g$ function must not be necessarily a polynomial. In fact one can use any function $g_i(x)$, for which holds $g_i(x)\in C^3[x_{i-1/2},x_{i+1/2}]$ for all $i$-cells.


\section{3rd Order Interpolation Polynomials}
\label{sec:IP}

Three data points are sufficient for constructing a unique quadratic interpolation function with the Ansatz function 
%
\begin{equation}
	%
	p_i(x)=a_i+b_i(x-x_i)+\frac{c_i}{2}(x-x_i)^2.
	\label{eq:AnsatzPolynomial}
	%
\end{equation}
%
To derive the coefficients $a_i$, $b_i$ and $c_i$ we impose the conditions set in Lemma 1. Therefore the quadratic function $p_i(x)$, integrated over the cell $C_i = [x_{i-1/2},x_{i+1/2}]$ has to recover the cell average $\bar{u}$ itself and its left and right derivatives located at the cell interfaces have to be approximated to second order $\mathcal{O}(\Delta x^2)$.

Given the set of cell averages ${\bar{u}_{i-1}},\bar{u}_{i},\bar{u}_{i+1}$ we require
%
\begin{align}
	%
	\frac{1}{\Delta x} \int_{C_i} p_i(x)dx =& \bar{u}_i, \\
	p'_i(x_i-\Delta x/2) =& d_1 =  \frac{\bar{u}_{i+1}-\bar{u}_{i}}{\Delta x}, \\
	p'_i(x_i+\Delta x/2) =& d_2 =  \frac{\bar{u}_{i}-\bar{u}_{i-1}}{\Delta x}.
	%
\end{align}
%
The above 
%
\begin{align}
	%
	a_i=& \bar{u}_i -\frac{1}{24}(\bar{u}_{i+1}- \bar{u}_i+\bar{u}_{i-1}), \\
	b_i=& \frac{1}{2\Delta x} (\bar{u}_{i+1} -\bar{u}_{i-1}), \\
	c_i=& \frac{1}{\Delta x^2} (\bar{u}_{i+1}- \bar{u}_i+\bar{u}_{i-1}).
	%
\end{align}
%
Evaluating the polynomial (\ref{eq:AnsatzPolynomial}) at the cell boundaries yields
%
\begin{align}
	%
	\hat{u}_{i+\frac{1}{2}}^{(-)}\equiv p_i(x_{i+1/2})=&
	-\frac{1}{6}\bar{u}_{i-1}
	+\frac{5}{6}\bar{u}_{ i }
	+\frac{1}{3}\bar{u}_{i+1}, \\
	\hat{u}_{i-\frac{1}{2}}^{(+)}\equiv p_i(x_{i-1/2})=&
	\,\,\frac{1}{3}\bar{u}_{i-1}
	+\frac{5}{6}\bar{u}_{ i }  
	-\frac{1}{6}\bar{u}_{i+1}.
	%
\end{align}
%
The idea of high-order polynomial reconstruction goes back to \cite{VanLeer1974II}. Van Leer published in a series of papers different local reconstructions functions including the quadratic interpolation. The schemes developed by Van Leer are second order accurate  on smooth regions of the solution, yet does not produce any spurious oscillations at jump discontinuities. Such methods, are now known as \emph{shock capturing schemes}.


\section{Constrained Interpolation Polynomials}
\label{sec:CIP}

Consider a scalar conservation law of the form
%
\begin{linenomath}
\begin{equation}
	%
	\pd{f}{t} + \upsilon \pd{f}{x} = 0, 
	\quad\text{for } x\in\Omega\in\mathbb{R},t\geqslant0, 
	\label{eq:scalarTransport}
	%
\end{equation}
\end{linenomath}
%
where $f:f(x,t)$ indicates a scalar field define in the domain $\Omega$ and to be transported in time. $\upsilon(x,t)\in\mathbb{R}$ is the characteristic advection speed. Let us now consider denote $g=\pd{f}{x}$ as the slope of the scalar field $f$. 

Taking the derivative of equation (\ref{eq:scalarTransport}) with respect to $x$ yields
%
\begin{linenomath}
\begin{equation}
	%
	\pd{}{x}\left(\pd{f}{t} + \upsilon \pd{f}{x}\right) = 
	\pd{}{x}\left(\pd{f}{t} \right) + 
	\pd{\upsilon}{x} \left( \pd{f}{x} \right) + 
	\upsilon \pd{}{x} \left( \pd{f}{x} \right) = 0,
	%
\end{equation}
\end{linenomath}
%
under the assumption of a constant velocity field $\upsilon$, the above result simply reduces to 
%
\begin{linenomath}
\begin{equation}
	%
	\pd{g}{t} + \upsilon \pd{g}{x} = 0
	\label{eq:slopeTransport}
	%
\end{equation}
\end{linenomath}
%
if both values of $f$ and $g$ are given at the cell boundaries, the profile of every cell can be reconstructed by an interpolation using a cubic polynomial 
%
\begin{linenomath}
\begin{equation}
	%
	F(x) = a \left(x-x_{i-\frac{1}{2}} \right)^3 + 
		   b \left(x-x_{i-\frac{1}{2}} \right)^2 + 
		   c \left(x-x_{i-\frac{1}{2}} \right) + d
	%
\end{equation}
\end{linenomath}
%
where $a$, $b$, $c$ and $d$ are to be defined under the constrains 
%
\begin{linenomath}
\begin{align}
	%
	F(x_{i-\frac{1}{2}}) = f_{i-\frac{1}{2}}, \\
	F(x_{i+\frac{1}{2}}) = f_{i+\frac{1}{2}}, \\
	\pd{F}{x}|_{x_{i-\frac{1}{2}}} = g_{i-\frac{1}{2}}, \\
	\pd{F}{x}|_{x_{i+\frac{1}{2}}} = g_{i+\frac{1}{2}}, \\
	F(x_i) = f(x_i) = f_i, \\
	\pd{F}{x}|_{x_i} = g(x_i) = g_i,
	%
\end{align}
\end{linenomath}
% 
however notice that under the following constrains the present scheme cannot guarantee conservation of cell average quantities $f_i$.

\section{Conservative semi-Lagrangian Constrained Interpolation Polynomials}
\label{sec:CIP-CSL}

Consider a scalar conservation law of the form
%
\begin{linenomath}
\begin{equation}
	%
	\pd{f}{t} + \pd{(\upsilon f)}{x} = 0, 
	\quad\text{for } x\in\Omega\in\mathbb{R},t\geqslant0, 
	\label{eq:scalarTransport2}
	%
\end{equation}
\end{linenomath}
%
where $f:f(x,t)$ indicates a scalar field define in the domain $\Omega$ and to be transported in time. $\upsilon(x,t)\in\mathbb{R}$ is the characteristic advection speed. Let us now consider denote $h=\int f\,dx$ and integrated scalar field of $f$.

Integrating equation (\ref{eq:scalarTransport}) with respect to $x$ yields
%
\begin{linenomath}
\begin{equation}
	%
	\int \left(\pd{f}{t} + \pd{(\upsilon f)}{x}\right) dx = 
	\pd{}{t} \int f dx + 
	\int  \pd{(\upsilon f)}{x} dx = 0
	%\pd{}{t} \int f dx + 
	%\int f \pd{\upsilon}{x} dx + \int \upsilon \pd{f}{x} dx = 0
	%
\end{equation}
\end{linenomath}
%
under the assumption of a constant velocity field $\upsilon$, the above result simply reduces to 
%
\begin{linenomath}
\begin{equation}
	%
	\pd{h}{t} + \upsilon \pd{h}{x} = 0
	\label{eq:integralTransport}
	%
\end{equation}
\end{linenomath}
%
since both equations (\ref{eq:scalarTransport}) and (\ref{eq:integralTransport}) are the same, the CIP procedure is applied to the pair $\int f dx$ and $f$ instead of $f$ and $\pd{f}{x}$. Using this analogy,  $h$ is defined as continuous function defined in the physical space of every cell as 
%
\begin{linenomath}
\begin{equation}
	%
	h(x) = \int_{x_{i-\frac{1}{2}}}^{x} f(\xi) d\xi,
	%
\end{equation}
\end{linenomath}
%
The profile of every cell can be reconstructed by an interpolation using a cubic polynomial of the form
%
\begin{linenomath}
\begin{equation}
	%
	h(x) = a \left(x-x_{i-\frac{1}{2}} \right)^3 + 
		   b \left(x-x_{i-\frac{1}{2}} \right)^2 + 
		   c \left(x-x_{i-\frac{1}{2}} \right) 
	%
\end{equation}
\end{linenomath}
%
where $a$, $b$ and $c$ are to be defined under the constrains 
%
\begin{linenomath}
\begin{align}
	%
	h(x_{i-\frac{1}{2}}) &= 0, \\
	h(x_{i+\frac{1}{2}}) &= \bar{f}_i \Delta x, \\
	\pd{h}{x}|_{x_{i-\frac{1}{2}}} &= f_{i-\frac{1}{2}}, \\
	\pd{h}{x}|_{x_{i+\frac{1}{2}}} &= f_{i+\frac{1}{2}}, \\
	\pd{h}{x}|_{x_i} = &f(x_i) = f_i, 
	%
\end{align}
\end{linenomath}
%
notice that $\bar{f_i}$ denotes the cell average and should not be confused with point value $f_i$. The cell average is here defined as,
%
\begin{linenomath}
\begin{equation}
	%
	\frac{1}{\Delta x_i} \int_{x_{i-\frac{1}{2}}}^{x_{i+\frac{1}{2}}} f^{n} dx = \bar{f_i^n}.
	%
\end{equation}
\end{linenomath}

\section{Conservative semi-Lagrangian Constrained Interpolation non-Polynomials}
\label{sec:CInonP-CSL}

\subsection{Rational Functions}

\dots (comming soon)

\subsection{Local Double Logarithmic Reconstruction}

The reconstruction functions is this cell has the form
%
\begin{linenomath}
\begin{equation}
	%
	r_0(x) \approx A + B\log(x+C)+ D\log(x+E),
	%
\end{equation}
\end{linenomath}
%
where $x\in C_0$ so that 
%
\begin{linenomath}
\begin{equation}
	%
	\phi_0(x) = 
	-\frac{c}{a} h\log\left[ x-x_0-\frac{h}{2}(\frac{2}{a}-1) \right] 
	-\frac{d}{b} h\log\left[ x-x_0-\frac{h}{2}(\frac{2}{b}-1) \right],
	%
\end{equation}
\end{linenomath}
%
where $a$, $b$, $c$ and $d$ are parameters to be determined, and let 
%
\begin{linenomath}
\begin{equation}
	%
	r_0 = \upsilon_0 + \phi_0(x) - \frac{1}{h} \int_{C_0} \phi_0(\xi) d\xi,
	\label{eq:LogarithmicReconstruction}
	%
\end{equation}
\end{linenomath}
%
be the reconstructing function. 
%
The reach third order we impose the conditions set in Lema 1, given $\upsilon_0$, the cell average of the unknown function in $C_0$ and $d_1$ and $d_2$ second order approximations to the left and right derivatives, we require
%
\begin{align}
	%
	\frac{1}{h} \int_{C_0} r_0(x)dx =& \upsilon_0, \\
	\pd{r_0}{x}\left(x_0-\frac{h}{2}\right) =&\,d_1, \\
	\pd{r_0}{x}\left(x_0+\frac{h}{2}\right) =&\,d_2.
	%
\end{align}
%
The reconstruction function (\ref{eq:LogarithmicReconstruction}) preserves the cell averages $\upsilon_0$ by construction. The avobe conditions lead to the following pair of equations
%
\begin{align}
	%
	c+d =& d_1, \\
	\frac{d(a-1)+c(b-1)}{(a-1)(b-a)} =& d_2.
	%
\end{align}
%
Thus the following four parameters to satisfy two equations. We solve for tow parameters, leaving the others undetermined for later use. Solving for $c$ and $d$ yields,
%
\begin{align}
	%
	c=& \,\frac{(a-1)(d_2(1-b)-d_1)}{b-a},\\
	d=& \,d_1-c.
	%
\end{align}
%
The reconstruction should be symmetric (to the extent allowed by the ansatz function) in the sense that if $r'_0(x_0-h/2) = -r'_0(x_0+h/2)$, then $r'(x_0)=0$. This can be realized by setting $d_1=-d_2$ and solving for $b$. We get
%
\begin{equation}
	%
	\pd{r_0(x_0)}{x}|_{d_1=-d_2} = 2d_2\frac{ab-a-b}{(a-2)(a-1)} = 0,
	%
\end{equation}
%
that is,
%
\begin{equation}
	%
	b = \frac{a}{a-1}.
	%	
\end{equation}
%
Notice that $a$ and $b$ are variables such that $a<1$ and $b<1$ and their range goes as ${a,b}\in\,(-\infty,1)$.  Also notice that $a=b$ only occurs when $a=b=0$.

Using the Marquina's concept of local variation boundness \citep{Artebrant2006}, variable $a$ is then defined as
%
\begin{equation}
	%
	a = (1-tol)\left( 1+tol-\frac{2|d_1|^{q}|d_2|^{q}+tol}{|d_1|^{2q}+|d_2|^{2q}+tol} \right).
	%
\end{equation}
%
where $d_1$ and $d_2$ are first order slopes approximations at the cell boundary. Here, $tol=0.1 h^q$ is a tolerance parameter defined in terms of constant $q=1.4$ and the cell size $h=(x_{i+1/2}-x_{i-1/2})$.

%% There might be applications where a different $q$ is preferred. But in, however, changing the exponent form is not recomended

Using directly the reconstruction function we then obtain
%
\begin{equation}
	%
	r_0(x_0\pm h/2) = \upsilon_0 ch\eta^{\pm}(a) + dh\eta^{\pm}(b),
	%
\end{equation}
%
where
%
\begin{equation}
	%
	\eta^{+}(t) = -\frac{log(1-t)+t}{t^2},
	%
\end{equation}
%
and
%
\begin{equation}
	%
	\eta^{+}(t) = \frac{(t-1)log(1-t)-t}{t^2}.
	%
\end{equation}
%
Notice in the above that we can exploit the fact 
%
\begin{equation}
	%
	\eta^{\pm} \left( b=\frac{a}{a-1}\right) = (a-1)\eta^{\pm}(a).
	%
\end{equation}
%
The functions $\eta^{\pm}$ have a removable singularity at zero by putting $\eta^{\pm} = \pm \frac{1}{2}$. 
 
\section{Concluding Remarks}
\label{sec:Conclusions}

% Begin the Bibliography
\bibliographystyle{humannat}
\bibliography{CIP.bib} 

\end{document}
\documentclass[10pt,a4paper]{article}
\usepackage{latexsym}
\usepackage{amsmath,amsfonts,amssymb,amscd,amsthm,xspace}  
\usepackage{natbib}

% Page length commands go here in the preamble
\setlength{\oddsidemargin}{0.10in}	% Left margin of 1 in + 0 in = 1 in
\setlength{\textwidth}{6.50in}		% Right margin of 8.5 in - 1 in - 6.5 in = 1 in
\setlength{\topmargin}{-0.75in}		% Top margin of 2 in -0.75 in = 1 in
\setlength{\textheight}{9.2in} 		% Lower margin of 11 in - 9 in - 1 in = 1 in

% For Theorems and Definitions
\newtheorem{theorem}{Theorem}
\newtheorem{definition}{Definition}

% Manuel's Macros
\renewcommand{\u}{\boldsymbol{\upsilon}} % for vector upsilon
\newcommand{\p}{\boldsymbol{p}} 		% for vector p
\renewcommand{\r}{\boldsymbol{r}} 		% for vector r
\newcommand{\x}{\boldsymbol{x}} 		% for vector x
\renewcommand{\c}{\boldsymbol{c}} 		% for vector c
\newcommand{\vecxi}{\boldsymbol{\xi}}	% for vector xi
\DeclareMathOperator{\Sample}{Sample}
\let\vaccent=\v % rename builtin command \v{} to \vaccent{}
\renewcommand{\v}[1]{\ensuremath{\mathbf{#1}}} % for vectors
\newcommand{\gv}[1]{\ensuremath{\mbox{\boldmath$ #1 $}}} 
% for vectors of Greek letters
\newcommand{\uv}[1]{\ensuremath{\mathbf{\hat{#1}}}} % for unit vector
\newcommand{\abs}[1]{\left| #1 \right|} % for absolute value
\newcommand{\avg}[1]{\left< #1 \right>} % for average
\let\underdot=\d % rename builtin command \d{} to \underdot{}
\renewcommand{\d}[2]{\frac{d #1}{d #2}} % for derivatives
\newcommand{\dx}[1]{\frac{d}{d #1}} % for derivatives without specifying the derived function
\newcommand{\dd}[2]{\frac{d^2 #1}{d #2^2}} % for double derivatives
\newcommand{\pd}[2]{\frac{\partial #1}{\partial #2}} 
% for partial derivatives
\newcommand{\pdd}[2]{\frac{\partial^2 #1}{\partial #2^2}} 
% for double partial derivatives
\newcommand{\pdc}[3]{\left( \frac{\partial #1}{\partial #2}
 \right)_{#3}} % for thermodynamic partial derivatives
\newcommand{\ket}[1]{\left| #1 \right>} % for Dirac bras
\newcommand{\bra}[1]{\left< #1 \right|} % for Dirac kets
\newcommand{\braket}[2]{\left< #1 \vphantom{#2} \right|
 \left. #2 \vphantom{#1} \right>} % for Dirac brackets
\newcommand{\matrixel}[3]{\left< #1 \vphantom{#2#3} \right|
 #2 \left| #3 \vphantom{#1#2} \right>} % for Dirac matrix elements
\newcommand{\grad}[1]{\gv{\nabla} #1} % for gradient
\newcommand{\gradx}[2]{\gv{\nabla}_{#2} #1} % for gradient
\let\divsymb=\div % rename builtin command \div to \divsymb
\renewcommand{\div}[1]{\gv{\nabla} \cdot #1} % for divergence
\newcommand{\divx}[2]{\gv{\nabla}_{#2} \cdot #1} % for divergence v2
\newcommand{\curl}[1]{\gv{\nabla} \times #1} % for curl

% Spacing and Indentation
\renewcommand{\baselinestretch}{1.5} % 1.5 denotes double spacing. Changing it will change the spacing
%\setlength{\parindent}{0in} 

%%%%%%%%%%%%%%%%%%%
% Document begins!
%%%%%%%%%%%%%%%%%%%

\begin{document}
\title{Conservative Semi-Lagrangian Reconstructions}
\author{Oleg Kravchenko\textsuperscript{1} \& Manuel Diaz* \\ 
\small \textsuperscript{1} Department of Nuclear Physics, University of Examples, Moscow, Russia. \\[-3mm]
\small * Institute of Applied Mechanics, National Taiwan University, Taiwan R.O.C.}
\date{\today}

\maketitle

\abstract{A class of high-order constrained reconstructions, in the context of finite difference methods for scalar hyperbolic conservation laws, are presented and numerically studied in the presence of local extrema, or shock solutions.
Although these reconstructions can use large time steps given their semi-Lagrangian in nature, a conservative formulation of the conservation is here considered. The numerical one-dimensional implementations in Matlab are reported to provide a simple follow up of the formulations here versed. Application to the nonlinear Boltzmann-BGK equation to model the dynamics of inert gas flows in near continuum regimes are explored.}

\section{Introduction}

%%%%%%%%%%%%%%
% NOTES:
%%%%%%%%%%%%%%
%
% **Note that nothing here right now if considered final. The purpose of this document is simply to serve as an enviroment (or a very orginized board) meant to collaborate/play/explore/test CIP formulations and any new idea that comes in time. Manuel Diaz. 2015/07/28.
%
% **Use "rsync -r /home/manuel/Dropbox/Apps/Texpad/Paper3/CIP/ /home/manuel/github/CIP/Tex/" to sincronize the Dropbox version with the GitHub version. (Debian 7 only!) Manuel Diaz. 2015/07/28.
%
%

% Constext of the problem/challenge/idea
The semi-Lagrangian method is a numerical solution technique for the partial differential equations describing the advection process. It accounts for the Lagrangian nature of the transport process but, at the same time, it allows to work on a fixed computational grid. Starting from the first proposals to in the meteorological literature, which focused on the advection of vorticity in simplified models of large scale flow, it has developed into a mature discretization approach for the complete equations of atmospheric flows. The semi-Lagrangian is also related (and, in some cases, entirely equivalent) to similar methods developed in other modelling communities, such as for example the modified method of characteristics, the Eulerian-Lagrangian method, the characteristic Galerkin method and the constrained interpolation polynomial methods. This last, have been proved to achieve excellent results for scalar conservation laws.
%
% I'll change the above. I took it from: http://www1.mate.polimi.it/~bonavent/Homepage_Luca_file/lezlag000.pdf

% What has been done?
Much work has been done in the text of semi-lagrangian constrained reconstructions with polynomials, here we point some of our observations:
\begin{itemize}
	\item \cite{Yabe1991} introduced the initial concept of cubic interpolation polynomial (CIP) for the evolution of scalar advection equations considering the also the evolution of the cell averages slope information. This simple idea has been proved to be workable but conservation cannot be guaranteed.
	\item Two years later \cite{Yabe2001}, introduced a new reformulation of their CIP algorithm which now considers the cell numerical-mass conservation as a constrain and a reconstruction of a flux as and integral function is introduced in this semi-Lagrangian formulation.
	\item Conservative formulation of CIP methods where first introduced by \cite{Tanaka2000} and \cite{Xiao2001}.
	\item Extention to multi-dimensional formulation of the CIP algorithms where reported by \cite{Nakamura2001}.
\end{itemize}

% What the solution procedure/proposal?
It is interesting to point out the conceptual similarities of CIP-CSL schemes in \cite{LiXiao2007} with the \emph{flux reconstruction} (FR) schemes by \cite{Huynh2007}. As both formulation achieve successfully high-order and accurate representations. However a clear advantage of the CIP-CSL is the simple incorporation of slope limiters in to the formulation which in the context of FR schemes breaks the abstraction of correction functions. Moreover, although their success the numerical experience of these methods reveals that both implementations require bast computational resources as their stability conditions can only allow small time steps. 
The above is also result of the fact that CIP-CSL and FR are closer to an Eulerian description of flow conditions, see for example \cite{LiXiao2009}.

It is found that CIP literature is rich in ideas and strategies to ensure the conservative properties of the method, improve their accuracy or deal with the oscillations that arise near local extrema. We notice that CIP implementation with rational functions by \cite{Xiao1996}. Here the use of such function instead of polynomials led to limiter free CIP second-order implementation that can deal with discontinuities in the solution in a straight manner. This observation resonates with the \emph{local dual logarithmic reconstructions} (LDLR) by \cite{Artebrant2006}. Which led us to consider a non-polynomial reconstructions to formulate new semi-Lagrangian reconstruction that can handle discontinuities that arise with the solutions in a much straight and computationally economic manner.

% How is this document organized?
The present document is , section \ref{sec:CIP} introduces the basic formulation of CIP by \cite{Yabe1991}, followed by its modification to a fully conservative semi-Lagrangian scheme in section \ref{sec:CIP-CSL}. In section \ref{sec:CInonP-CSL} the concept of constrained interpolation is considered with a non-polynomial function. Lastly, some concluding remarks are given section \ref{sec:Conclusions}.

\section{Constrained Interpolation Polynomials}
\label{sec:CIP}

Consider a scalar conservation law of the form
%
\begin{equation}
	%
	\pd{f}{t} + \upsilon \pd{f}{x} = 0, 
	\quad\text{for } x\in\Omega\in\mathbb{R},t\geqslant0, 
	\label{eq:scalarTransport}
	%
\end{equation}
%
where $f:f(x,t)$ indicates a scalar field define in the domain $\Omega$ and to be transported in time. $\upsilon(x,t)\in\mathbb{R}$ is the characteristic advection speed. Let us now consider denote $g=\pd{f}{x}$ as the slope of the scalar field $f$. 

Taking the derivative of equation (\ref{eq:scalarTransport}) with respect to $x$ yields
%
\begin{equation}
	%
	\pd{}{x}\left(\pd{f}{t} + \upsilon \pd{f}{x}\right) = 
	\pd{}{x}\left(\pd{f}{t} \right) + 
	\pd{\upsilon}{x} \left( \pd{f}{x} \right) + 
	\upsilon \pd{}{x} \left( \pd{f}{x} \right) = 0,
	%
\end{equation}
%
under the assumption of a constant velocity field $\upsilon$, the above result simply reduces to 
% 
\begin{equation}
	%
	\pd{g}{t} + \upsilon \pd{g}{x} = 0
	\label{eq:slopeTransport}
	%
\end{equation}
%
if both values of $f$ and $g$ are given at the cell boundaries, the profile of every cell can be reconstructed by an interpolation using a cubic polynomial 
%
\begin{equation}
	%
	F(x) = a \left(x-x_{i-\frac{1}{2}} \right)^3 + 
		   b \left(x-x_{i-\frac{1}{2}} \right)^2 + 
		   c \left(x-x_{i-\frac{1}{2}} \right) + d
	%
\end{equation}
%
where $a$, $b$, $c$ and $d$ are to be defined under the constrains 
%
\begin{align}
	%
	F(x_{i-\frac{1}{2}}) = f_{i-\frac{1}{2}}, \\
	F(x_{i+\frac{1}{2}}) = f_{i+\frac{1}{2}}, \\
	\pd{F}{x}|_{x_{i-\frac{1}{2}}} = g_{i-\frac{1}{2}}, \\
	\pd{F}{x}|_{x_{i+\frac{1}{2}}} = g_{i+\frac{1}{2}}, \\
	F(x_i) = f(x_i) = f_i, \\
	\pd{F}{x}|_{x_i} = g(x_i) = g_i,
	%
\end{align}
% 
however notice that under the following constrains the present scheme cannot guarantee conservation of cell average quantities $f_i$.

\section{Conservative semi-Lagrangian Constrained Interpolation Polynomials}
\label{sec:CIP-CSL}

Consider a scalar conservation law of the form
%
\begin{equation}
	%
	\pd{f}{t} + \pd{(\upsilon f)}{x} = 0, 
	\quad\text{for } x\in\Omega\in\mathbb{R},t\geqslant0, 
	\label{eq:scalarTransport2}
	%
\end{equation}
%
where $f:f(x,t)$ indicates a scalar field define in the domain $\Omega$ and to be transported in time. $\upsilon(x,t)\in\mathbb{R}$ is the characteristic advection speed. Let us now consider denote $h=\int f\,dx$ and integrated scalar field of $f$.

Integrating equation (\ref{eq:scalarTransport}) with respect to $x$ yields
%
\begin{equation}
	%
	\int \left(\pd{f}{t} + \pd{(\upsilon f)}{x}\right) dx = 
	\pd{}{t} \int f dx + 
	\int  \pd{(\upsilon f)}{x} dx = 0
	%\pd{}{t} \int f dx + 
	%\int f \pd{\upsilon}{x} dx + \int \upsilon \pd{f}{x} dx = 0
	%
\end{equation}
%
under the assumption of a constant velocity field $\upsilon$, the above result simply reduces to 
% 
\begin{equation}
	%
	\pd{h}{t} + \upsilon \pd{h}{x} = 0
	\label{eq:integralTransport}
	%
\end{equation}
%
since both equations (\ref{eq:scalarTransport}) and (\ref{eq:integralTransport}) are the same, the CIP procedure is applied to the pair $\int f dx$ and $f$ instead of $f$ and $\pd{f}{x}$. Using this analogy,  $h$ is defined as continuous function defined in the physical space of every cell as 
%
\begin{equation}
	%
	h(x) = \int_{x_{i-\frac{1}{2}}}^{x} f(\xi) d\xi,
	%
\end{equation}
%
The profile of every cell can be reconstructed by an interpolation using a cubic polynomial of the form
%
\begin{equation}
	%
	h(x) = a \left(x-x_{i-\frac{1}{2}} \right)^3 + 
		   b \left(x-x_{i-\frac{1}{2}} \right)^2 + 
		   c \left(x-x_{i-\frac{1}{2}} \right) 
	%
\end{equation}
%
where $a$, $b$ and $c$ are to be defined under the constrains 
%
\begin{align}
	%
	h(x_{i-\frac{1}{2}}) &= 0, \\
	h(x_{i+\frac{1}{2}}) &= \bar{f}_i \Delta x, \\
	\pd{h}{x}|_{x_{i-\frac{1}{2}}} &= f_{i-\frac{1}{2}}, \\
	\pd{h}{x}|_{x_{i+\frac{1}{2}}} &= f_{i+\frac{1}{2}}, \\
	\pd{h}{x}|_{x_i} = &f(x_i) = f_i, 
	%
\end{align}
%
notice that $\bar{f_i}$ denotes the cell average and should not be confused with point value $f_i$. The cell average is here defined as,
%
\begin{equation}
	%
	\frac{1}{\Delta x_i} \int_{x_{i-\frac{1}{2}}}^{x_{i+\frac{1}{2}}} f^{n} dx = \bar{f_i^n}.
	%
\end{equation}

\section{Conservative semi-Lagrangian Constrained Interpolation non-Polynomials}
\label{sec:CInonP-CSL}



\section{Concluding Remarks}
\label{sec:Conclusions}

% Begin the Bibliography
\bibliographystyle{humannat}
\bibliography{CIP.bib} 

\end{document}